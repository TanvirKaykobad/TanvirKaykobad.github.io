
% ===========================================================================
% Title:
% ---------------------------------------------------------------------------
% to create Type I fonts type "dvips -P cmz -t letter <filename>"
% ===========================================================================
\documentclass[11pt]{article}       %--- LATEX 2e base
\usepackage{latexsym}               %--- LATEX 2e base
%---------------- Wide format -----------------------------------------------
\textwidth=6in \textheight=9in \oddsidemargin=0.25in
\evensidemargin=0.25in \topmargin=-0.5in
%--------------- Def., Theorem, Proof, etc. ---------------------------------
\newtheorem{definition}{Definition}
\newtheorem{theorem}{Theorem}
\newtheorem{lemma}{Lemma}
\newtheorem{corollary}{Corollary}
\newtheorem{property}{Property}
\newtheorem{observation}{Observation}
\newtheorem{fact}{Fact}
\newenvironment{proof}           {\noindent{\bf Proof.} }%
                                 {\null\hfill$\Box$\par\medskip}
%--------------- Algorithm --------------------------------------------------
\newtheorem{algX}{Algorithm}
\newenvironment{algorithm}       {\begin{algX}\begin{em}}%
                                 {\par\noindent --- End of Algorithm ---
                                 \end{em}\end{algX}}
\newcommand{\step}[2]            {\begin{list}{}
                                  {  \setlength{\topsep}{0cm}
                                     \setlength{\partopsep}{0cm}
                                     \setlength{\leftmargin}{0.8cm}
                                     \setlength{\labelwidth}{0.7cm}
                                     \setlength{\labelsep}{0.1cm}    }
                                  \item[#1]#2    \end{list}}
                                 % usage: \begin{algorithm} \label{xyz}
                                 %        ... \step{(1)}{...} ...
                                 %        \end{algorithm}
%--------------- Figures ----------------------------------------------------
\usepackage{graphicx}

\newcommand{\includeFig}[3]      {\begin{figure}[htb] \begin{center}
                                 \includegraphics
                                 [width=4in,keepaspectratio] %comment this line to disable scaling
                                 {#2}\caption{\label{#1}#3} \end{center} \end{figure}}
                                 % usage: \includeFig{label}{file}{caption}


% ===========================================================================
\begin{document}
% ===========================================================================

% ############################################################################
% Title
% ############################################################################

\title{--- Your Project Title ---}


% ############################################################################
% Author(s) (no blank lines !)
\author{
% ############################################################################
John Doe\\
School of Computer Science\\
Carleton University\\
Ottawa, Canada K1S 5B6\\
{\em John-Doe@scs.carleton.ca}
% ############################################################################
} % end-authors
% ############################################################################

\maketitle

% ############################################################################
% Abstract
% ############################################################################
\begin{abstract}
Give a brief overview of what you have achieved.
\end{abstract}


% ############################################################################
\section{Introduction} \label{intro}
% ############################################################################

Introduce your project topic (start from parallel computing in
general and lead to your particular topic). Describe your project
goals. Describe what you have achieved in your project. Outline
the structure of your paper. (In Section~\ref{litrev}, we will
review the relevant literature. Section~\ref{projrep} will present
the results of our project. In Section~\ref{subsect1}, ...
Section~\ref{concl} concludes the paper.


% ############################################################################
\section{Literature Review} \label{litrev}
% ############################################################################

Give an overview of the relevant literature. Cite all relevant
papers, like \cite{DEL07}, \cite{PD07}, \cite{DER07}, \cite{LDR07},
\cite{DLX06}, \cite{CDE06}, and \cite{DFL06}. Outline for each paper
the relevant results in relation to your project. Make sure that you
don't just list all relevant papers in random order. Devise a scheme
to group papers by subject. The goal of this section is to present
to the reader the state-of-the-art in the field selected for your
project.

% ############################################################################
\section{Project Report} \label{projrep}
% ############################################################################

Present the results of your project. Add subsections as appropriate...

% ----------------------------------------------------------------------------
\subsection{Subsection 1} \label{subsect1}
% ----------------------------------------------------------------------------

...

% ----------------------------------------------------------------------------
\subsection{Subsection 2} \label{subsect2}
% ----------------------------------------------------------------------------

...

% ----------------------------------------------------------------------------
\subsection{Subsection 3} \label{subsect3}
% ----------------------------------------------------------------------------

...

You can also have figures in your paper. Figure~\ref{fig1} is a
typical example of an experimental evaluation result. Such graphs
are ususally created with GnuPlot. Figure~\ref{fig2} is an example
of a drawing created with {\em mdraw} or {\em epsfig}.

% usage: \includeFig{label}{file}{caption}

\includeFig{fig1}{Figures/figure-1}{Measured Running Times
Of Some Unknown Algorithm Implementation}

\includeFig{fig2}{Figures/figure-2}{XYZ and Hilbert Packings}




% ############################################################################
\section{Conclusion} \label{concl}
% ############################################################################

The ``moral of the story'': What have we learned? What did we achieve?
What did we not achieve? What would we do better next time? Possibilities
for future research...

% ############################################################################
% Bibliography
% ############################################################################
\bibliographystyle{plain}
\bibliography{my-bibliography}     %loads my-bibliography.bib

% ============================================================================
\end{document}
% ============================================================================
