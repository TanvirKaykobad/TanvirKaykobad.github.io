
% ===========================================================================
% Title:
% ---------------------------------------------------------------------------
% to create Type I fonts type "dvips -P cmz -t letter <filename>"
% ===========================================================================
\documentclass[11pt]{article}       %--- LATEX 2e base
\usepackage{latexsym}               %--- LATEX 2e base
%---------------- Wide format -----------------------------------------------
\textwidth=6in \textheight=9in \oddsidemargin=0.25in
\evensidemargin=0.25in \topmargin=-0.5in
%--------------- Slide --------------------------------------------------
\newenvironment{slide}[1]        {\section{#1} \begin{itemize}}%
                                 {\end{itemize}}
                                 % usage: \begin{slide}{---SLIDE TITLE---}
                                 %        \item ...
                                 %        \item ...
                                 %        \item ...
                                 %        \end{slide}
%---------------Math--------------------------------------------------------
%\usepackage{mathtools}
%\DeclarePairedDelimiter{\ceil}{\lceil}{\rceil}

\usepackage{skmath}
\ExplSyntaxOn
\RenewDocumentCommand\log{oom}{%
  \IfNoValueTF{#1}
    {\ensuremath{\__skmath_log:\IfNoValueTF{#2}{}{\c_math_superscript_token{#2}}\__skmath_parens:n{#3}}}
    {\ensuremath{\__skmath_log:\c_math_subscript_token{#1}\IfNoValueTF{#2}{}{\c_math_superscript_token{#2}}\__skmath_parens:n{#3}}}%
}
\ExplSyntaxOff

% ===========================================================================
\begin{document}

% ############################################################################
% Title
% ############################################################################

\title{PRESENTATION OUTLINE: --- Accurately computing large floating-point numbers using parallel computing ---}


% ############################################################################
% Author(s) (no blank lines !)
\author{
% ############################################################################
Tanvir Kaykobad\\
School of Computer Science\\
Carleton University\\
Ottawa, Canada K1S 5B6\\
{\em tanvirkaykobad@cmail.carleton.ca}
% ############################################################################
} % end-authors
% ############################################################################

\maketitle

% ############################################################################
\begin{slide}{Parallel algorithms for summing floating point numbers}
\item Parallelism important for scaling
\item Most parallel algorithms are based on parallelising data pipelining and not on polylogarithmic running times.
\item Eatl introduced superaccumulator for efficient parallel algorithms for faithfully rounded floating point sum for exactly summing n numbers.

\item This paper computes sum of n numbers in $O(\log[][] {n})$ time using $n$ processors 
\item Sum can be computed in $O(\log[][2]{n}\log[][]{}\log[][]{}\log[][]{}C(X))$ time using $O(n\log[][]{}C(X))$ work
\end {slide}


% ############################################################################
% Bibliography
% ############################################################################
\bibliographystyle{plain}
\bibliography{my-bibliography}     %loads my-bibliography.bib

% ============================================================================
\end{document}
% ============================================================================
